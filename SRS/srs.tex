\documentclass{scrreprt}
\usepackage{listings}
\usepackage{underscore}
\usepackage[bookmarks=true]{hyperref}
\usepackage[utf8]{inputenc}
\usepackage[english]{babel}
\hypersetup{
    bookmarks=false,    % show bookmarks bar
    pdftitle={Software Requirement Specification},    % title
    pdfauthor={CES4020-Group5},                     % author
    pdfsubject={TeX and LaTeX},                        % subject of the document
    pdfkeywords={TeX, LaTeX, graphics, images}, % list of keywords
    colorlinks=true,       % false: boxed links; true: colored links
    linkcolor=blue,       % color of internal links
    citecolor=black,       % color of links to bibliography
    filecolor=black,        % color of file links
    urlcolor=purple,        % color of external links
    linktoc=page            % only page is linked
}%
\def\myversion{1.0 }
\def\mygroup{5 }
\def\myclass{CES4020 }
\date{}
%\title
\usepackage{hyperref}
\begin{document}

\begin{flushright}
    \rule{16cm}{5pt}\vskip1cm
    \begin{bfseries}
        \Huge{SOFTWARE REQUIREMENTS SPECIFICATION}\\
        \vspace{1.9cm}
        for\\
        \vspace{1.9cm}
        $Weather App$\\
        \vspace{1.9cm}
        \LARGE{\myclass}\\
        \vspace{1.9cm}
        Prepared by $Dave\hspace{1mm}Dorzback$\\$Abdullah\hspace{1mm}Alsayyar$\\$Michele\hspace{1mm}Prima$\\$Trelanah\hspace{1mm}McCalla$\\
        \vspace{1.9cm}
        Group\hspace{1mm}\mygroup\\
        \vspace{1.9cm}
        \today\\
    \end{bfseries}
\end{flushright}

\tableofcontents


%\chapter*{Revision History}

%\begin{center}
%    \begin{tabular}{|c|c|c|c|}
%        \hline
%	    Name & Date & Reason For Changes & Version\\
%        \hline
%	    21 & 22 & 23 & 24\\
%        \hline
%	    31 & 32 & 33 & 34\\
%        \hline
%    \end{tabular}
%\end{center}

\chapter{Introduction}

\section{Purpose}
The purpose of project is to design and build a fully functional weather desktop application. The application’s primary objective is to provide weather forecast data to users based on geographic location in a simple and intuitive manner. 

\section{Project Scope}
The application’s objective is to provide an easy and straightforward manner in which users can obtain weather forecasts for their city. New users will register accounts with the application, and the application’s database will be able to support up to 20 user accounts.  Moreover, the application will include two tiers of user accounts, Free Users and Premium Users, and offer different features based on users’ account type. The application’s focus will be on users in the United States, and the goal is to provide weather forecasts for every city within the country.					
					
\section{Acronyms, Abbreviations, and Definitions}
\begin{itemize}

\item Free Users(Free memberships) - users that have not purchased the app subscription. App functions and features will be limited for Free Users.

\item Premium Users(Paid memberships) - users that have purchased the app subscription. Premium Users will have access to all app features. 

\item Weather Calls - also listed as “calls.” Calls are sent Dark Sky API requests. Requests are used to update current weather or to get the projected forecast for a future date.

\item Graphic Depicting - small icons that correspond to the weather. For instance, on sunny days, the app would display a small sun icon, on cloudy days, there will be a cloud etc.
				
\end{itemize}

\section{Application Functionality}
This application's main function is to show weather information for a given US city. Users can search by city names. The app can show the weather for of the next seven days. In the free membership, users are able to search and store one location only. On the other hand, the paid membership allows users to store up to five locations. 	

\section{Operating Environment}
The application will be compatible on desktop or laptop computers running Windows 10. The application will utilize Dark Sky API to receive weather forecasts.	

\section{Design and Implementation Constraints}
$<$Describe any items or issues that will limit the options available to the 
developers. These might include: corporate or regulatory policies; hardware 
limitations (timing requirements, memory requirements); interfaces to other 
applications; specific technologies, tools, and databases to be used; parallel 
operations; language requirements; communications protocols; security 
considerations; design conventions or programming standards).$>$

\section{User Characteristics}
Users of the application will be limited to persons within the United States. Users must also have access to a computer that operates on Windows 10. There are no age or class restrictions on the application.
					
\section{Assumptions and Dependencies}
The Dark Sky API will be a dependency for the project as the application will pull weather data using this API. Requirements relating to the number of API calls are based on the assumption that constraints around the Dark Sky API will not change. If the Dark Sky API changes their limitations on the number of calls that can be made each day, this could affect these requirements. 		


\chapter{Requirements}

\section{Functional Requirements}

\subsection{Total Call Numbers}
This application will support up to 1000 calls per day. Total calls is the summation of all calls made by all accounts.	

\subsection{Search by City Name}
When searching for a location to display its weather, the app allows users to search using city names. This will make it easy for most users to use the app quickly enough when looking for a desired location.

\subsection{New Accounts}
Users that do not have an account will directed to make an account and choose to become a Free User or a Premium User.

\subsection{Location Changes}
Both Free and Premium Users will be able to remove saved locations.

\subsection{US Locations Only}
The application is limited to be used for cities and locations in the US. No other countries will be available for users.

\subsection{Navigation via Tabs}
The application’s current-day and 7-weather displays will use a tab design for navigation. The user will be able to navigate through the weather forecasts using tabs.

\subsection{Advance Weather Prediction}
While users can enjoy monitoring the weather momentarily, they will also enjoy a 7-day in advance weather prediction for their searched and stored cities.   	

\section{Preformance Requirements}

\section{Security Requirements}

\subsection{Password Hashing}
User passwords will be hashed with a simple but well-designed key stretching algorithm before being stored in the application’s database. This will reduce the likelihood of passwords being cracked and protect user data from security threats such as brute-force and dictionary attacks.

\section{Quality Requirements}

\subsection{Premium User Call Numbers}
Premium Users will be able to make up to 66 calls per day. After the call limit has been reached, the app will not respond to call requests.

\subsection{Free User Call Numbers}
Free Users will be able to make up to 33 calls per day. After the call limit has been reached, the app will not respond to call requests.

\subsection{Graphic Depicting}
In addition to having weather information displayed for searched cities, the app will provide graphic depicting through icons like a cloud indicating cloudy days, a sun for sunny days or thunders for stormy days etc. 

\subsection{Database Capacity}
The application’s database will be able to reliably support up to 20 different user accounts and all data associated with those accounts. 

\subsection{FAQ Page}
The application will include an FAQ or similar page that describes the application’s terms, its functionality, and any other information deemed relevant to users. The FAQ page will be easily accessible to users.

\subsection{Design Simplicity}
The application will be designed in such a way that after training, at least 90\% of users will be able to successfully navigate and understand the application. 

\subsection{Paid Membership Limitations}
The application will be compatible with desktop or laptop computers operating on Windows 10.

\section{External Interface Requirements}

\subsection{Hardware and Software Compatibility}
The app will be compatible with desktop or laptop computers operating on Windows 10.


%\section{Hardware Interfaces}

%\section{Software Interfaces}


\end{document}
